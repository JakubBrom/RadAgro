%% Generated by Sphinx.
\def\sphinxdocclass{report}
\documentclass[letterpaper,10pt,english]{sphinxmanual}
\ifdefined\pdfpxdimen
   \let\sphinxpxdimen\pdfpxdimen\else\newdimen\sphinxpxdimen
\fi \sphinxpxdimen=.75bp\relax

\PassOptionsToPackage{warn}{textcomp}
\usepackage[utf8]{inputenc}
\ifdefined\DeclareUnicodeCharacter
% support both utf8 and utf8x syntaxes
  \ifdefined\DeclareUnicodeCharacterAsOptional
    \def\sphinxDUC#1{\DeclareUnicodeCharacter{"#1}}
  \else
    \let\sphinxDUC\DeclareUnicodeCharacter
  \fi
  \sphinxDUC{00A0}{\nobreakspace}
  \sphinxDUC{2500}{\sphinxunichar{2500}}
  \sphinxDUC{2502}{\sphinxunichar{2502}}
  \sphinxDUC{2514}{\sphinxunichar{2514}}
  \sphinxDUC{251C}{\sphinxunichar{251C}}
  \sphinxDUC{2572}{\textbackslash}
\fi
\usepackage{cmap}
\usepackage[T1]{fontenc}
\usepackage{amsmath,amssymb,amstext}
\usepackage{babel}



\usepackage{times}
\expandafter\ifx\csname T@LGR\endcsname\relax
\else
% LGR was declared as font encoding
  \substitutefont{LGR}{\rmdefault}{cmr}
  \substitutefont{LGR}{\sfdefault}{cmss}
  \substitutefont{LGR}{\ttdefault}{cmtt}
\fi
\expandafter\ifx\csname T@X2\endcsname\relax
  \expandafter\ifx\csname T@T2A\endcsname\relax
  \else
  % T2A was declared as font encoding
    \substitutefont{T2A}{\rmdefault}{cmr}
    \substitutefont{T2A}{\sfdefault}{cmss}
    \substitutefont{T2A}{\ttdefault}{cmtt}
  \fi
\else
% X2 was declared as font encoding
  \substitutefont{X2}{\rmdefault}{cmr}
  \substitutefont{X2}{\sfdefault}{cmss}
  \substitutefont{X2}{\ttdefault}{cmtt}
\fi


\usepackage[Bjarne]{fncychap}
\usepackage[,numfigreset=3,mathnumfig]{sphinx}

\fvset{fontsize=\small}
\usepackage{geometry}


% Include hyperref last.
\usepackage{hyperref}
% Fix anchor placement for figures with captions.
\usepackage{hypcap}% it must be loaded after hyperref.
% Set up styles of URL: it should be placed after hyperref.
\urlstyle{same}


\usepackage{sphinxmessages}
\setcounter{tocdepth}{2}


\usepackage{mathtools}
\usepackage{inputenc}
\usepackage{siunitx}


\title{RadAgro - Dokumentace}
\date{Nov 09, 2020}
\release{0.1}
\author{Autoři: Jakub Brom a kol.}
\newcommand{\sphinxlogo}{\vbox{}}
\renewcommand{\releasename}{Release}
\makeindex
\begin{document}

\pagestyle{empty}
\sphinxmaketitle
\pagestyle{plain}
\sphinxtableofcontents
\pagestyle{normal}
\phantomsection\label{\detokenize{index::doc}}


Čekali opakovat tentokrát. Osvobozující, škytla mladou uraženě energický
spáchaná. Mnou nemá úhlů kolo s řezat hrůzy hořkostí že krvavé nedošlo
svlékněte babu. Per pasu tu takzvanou hafáček, místná němu mu hluchý kříž
okamžik třásly k ptá vina oč hygienickým vášeň řeč on mé výslechů. Slušné
jenom působilo míjí vševědoucí všech chvílích spíži bys nějakém cennou ‚milý
zklamáním vy ruce úředník drahocenný.


\chapter{Obsah:}
\label{\detokenize{index:obsah}}

\section{Popis programu RadAgro}
\label{\detokenize{description:popis-programu-radhydro}}\label{\detokenize{description::doc}}
Modul Urban Green SARCA je softwarový GIS nástroj vytvořený pro účely odhadu
depozice radionuklidu na povrchu vegetace a půdy, respektive ploch bez
vegetace v časné fázi radiační havárie. Modul umožňuje odhadnout kontaminaci
vegetačního krytu prostřednictvím satelitních snímků, podkladů o celkové
depozici radionuklidu a na základě informace o úhrnu srážek v průběhu
depozice radionuklidu v zájmovém území. Výpočet je tedy prováděn pro podmínky
suché i mokré depozice radionuklidu. Vzhledem k tomu, že modul umožňuje
výpočet jednotlivých proměnných s využitím podrobných satelitních dat, je
využití modulu vhodné zejména pro urbánní území a pro území, kde nejsou k
dispozici informace o jednotlivých typech zeleně (vegetace) nebo je
identifikace zeleně značně komplikovaná.

Modul je koncipován tak, aby minimalizoval množství vstupů a zároveň
poskytoval dostatečné množství výstupů důležitých pro následné rozhodování v
oblasti radiační ochrany urbánních území.


\subsection{Funkcionalita modulu Urban Green SARCA}
\label{\detokenize{description:funkcionalita-modulu-urban-green-sarca}}
Funkcionalita modulu vychází z programu SARCA (Brom et al. 2015), který je
nicméně určen pro analýzu kontaminace polních plodin a je založen na
modelování časových změn produkčních charakteristik plodin. Oproti tomu,
Urban Green SARCA využívá pro odhad produkčních charakteristik vegetace
dostupná satelitní data.

Výpočet jednotlivých výstupů lze rozdělit do dvou částí. V první části jsou
ze satelitních dat vypočteny produkční charakteristiky, tedy množství biomasy
a index listové plochy vegetace. Ve druhé části je vypočtena depozice
radioaktivního kontaminantu na povrchu vegetace a na ostatních površích. Na
základě vypočtených dat jsou dále hodnoceny kategorie radioaktivní
kontaminace vegetace (zeleně) podle stanovených referenčních úrovní
kontaminace (viz dále). Doplňkově je vypočtena vrstva intercepčního faktoru,
tedy relativní distribuce radionuklidu mezi zeleň a ostatní povrchy a
vypočtena je též hmotnostní kontaminace, včetně vyznačení nadlimitní
hmotnostní kontaminace zeleně. Způsoby výpočtu jednotlivých ukazatelů jsou
uvedeny dále.


\subsection{Výpočet výstupů modulu Urban Green SARCA}
\label{\detokenize{description:vypocet-vystupu-modulu-urban-green-sarca}}

\subsubsection{Množství biomasy}
\label{\detokenize{description:mnozstvi-biomasy}}
Při výpočtu množství biomasy (B; \(t.ha^{-1}\)) na dané ploše vycházíme ze
vztahu mezi množstvím zelené biomasy a jejím spektrálním projevem. Pro
vyjádření byl použit následující zjednodušený vztah pro odhad množství biomasy:
\begin{equation*}
\begin{split}B=50\cdot NDVI^{2.5},\end{split}
\end{equation*}
kde NDVI je normalizovaný rozdílový vegetační index (Normalized Difference
Vegetation Index; Rouse Jr et al., (1973)):
\begin{equation*}
\begin{split}NDVI=\frac{R_{NIR}-R_{RED}}{R_{NIR}+R_{RED}},\end{split}
\end{equation*}
kde \(R_{NIR}\) a \(R_{RED}\) jsou spektrální reflektance v blízké
inrfačervené (NIR; přibližně 800 nm) a v červené oblasti (RED; přibližně 670
nm) (rel.). Použitý model představuje hrubý odhad živé biomasy zeleně. Do
budoucna předpokládáme nahrazení uvedeného vztahu komplexnějším modelem.


\subsubsection{Index listové plochy}
\label{\detokenize{description:index-listove-plochy}}
Index listové plochy (bezrozm.) je počítán pomocí jednoduchého lineárního
vztahu mezi listovou plochou a spektrálním indexem NDVI:
\begin{equation*}
\begin{split}LAI=4.9\cdot NDVI-0.46\end{split}
\end{equation*}

\subsubsection{Kontaminace zeleně a půdy, intercepční faktor}
\label{\detokenize{description:kontaminace-zelene-a-pudy-intercepcni-faktor}}
Pro rozhodování o množství depozice radioaktivního materiálu na povrchu
porostu a povrchu půdy je vypočten intercepční faktor (rel.), který je
ukazatelem, jak velká frakce depozice zůstává na povrchu porostu. Hodnota
závisí na indexu listové plochy porostu a úhrnu srážek v průběhu depozice.
Podle Müllera a Pröhla (1993) lze intercepční frakci (faktor) depozice
radioizotopu fw v časné fázi radiační havárie vypočítat podle vzorce:
\begin{equation*}
\begin{split}f_{w}=\min\left[1;\frac{LAI\cdot k\cdot S\left(1-\mathrm{e^{-\frac{\ln2}{3S}R}}\right)}{R}\right]\end{split}
\end{equation*}
kde k je specifický faktor pro daný kontaminant (I: k = 0.5; Sr, Ba: k = 2;
Cs a ostatní radionuklidy: k = 1), S je tloušťka vodního filmu na rostlinách
(mm) a R je úhrn srážek (mm). Hodnota S je zpravidla 0,15 \textendash{} 0,3 mm se střední
hodnotou 0,2 mm (Pröhl, 2003). Výpočet depozice na povrchu rostlin vychází z
předpokladu, že depozice na povrchu rostlin je poměrnou částí celkové
depozice danou intercepčním faktorem:
\begin{equation*}
\begin{split}D_{biomasa}=D_{celk}\cdot f_{w}\end{split}
\end{equation*}
kde \(D_{biomasa}\) je měrná depozice radioizotopu na povrchu rostlin
\((Bq.m^{-2})\) a Dcelk je celková měrná radioaktivní depozice \((Bq
.m^{-2})\) zadávaná jako vstup do modelu. Měrná depozice radioizotopu na
povrchu půdy (Dpuda ; \(Bq.m^{-2}\)) je pak rozdílem mezi celkovou měrnou
depozicí a měrnou depozicí na povrchu porostu:
\begin{equation*}
\begin{split}D_{puda}=D_{celk}-D_{biomasa}\end{split}
\end{equation*}
Pokud jsou hodnoty vypočteného množství biomasy menší než 0,5 \(t
.ha^{-1}\), je vypočtena pouze měrná depozice radioaktivního materiálu na
povrchu půdy. Důvodem je minimální předpoklad možnosti odstranění biomasy.
Doplňkovou charakteristikou je hmotnostní kontaminace biomasy zeleně (Dhmot;
\(Bq.m^{-2}\)), která je vypočtena podle vztahu:
\begin{equation*}
\begin{split}D_{hmot}=\frac{D_{biomasa}}{B \cdot 0.1}\end{split}
\end{equation*}

\subsubsection{Referenční úrovně a  překročení hygienického limitu kontaminace biomasy}
\label{\detokenize{description:referencni-urovne-a-prekroceni-hygienickeho-limitu-kontaminace-biomasy}}
Území kontaminované radioaktivní depozicí je pro praktické účely rozděleno na
tři oblasti, v závislosti na stanovených referenčních úrovních. Rozdělení
sledovaného území do oblastí podle referenčních úrovní vychází z předpokladu,
že lze vymezit území, ve kterých kontaminace nepřekračuje stanovenou úroveň
dávkového příkonu nebezpečného pro obyvatelstvo a zvířata (hodnota 0), dále
území ve kterých lze provádět opatření za účelem radiační ochrany (hodnota 1)
a území, kde úroveň radioaktivní kontaminace, respektive dávkového příkonu
překračuje bezpečnou hranici pro další management (hodnota 2).
Pro referenční úrovně RU 0 a RU 2 není doporučeno odstranění biomasy za
účelem ochrany půdy. V prvním případě (RU 0) nepřesahuje kontaminace
stanovenou mez a nejsou ze předpokládána další rizika, zeleň a produkci
rostlinné biomasy je možné využít běžným způsobem, případně v omezené míře na
základě dalších postupů. Naopak v případě ploch zařazených do referenční
úrovně RU 2 existuje předpoklad nadlimitní radioaktivní kontaminace ploch a
možnost ohrožení zdraví pracovníků pověřených manipulací s nadzemní biomasou
rostlin. V rámci ploch zařazených do RU 1 lze předpokládat půdoochranný
význam vegetačního krytu, který lze za daných podmínek odstranit z půdního
povrchu. Limitem je zde množství živé nadzemní biomasy 0,5 \(t.ha^{-1}\),
kdy předpokládáme, že sklizeň menšího množství biomasy na danou plochu je již
neefektivní, případně technicky nemožná. Hranice referenčních úrovní lze
nastavit přímo v uživatelském rozhraní Urban Green SARCA. Přednastaveny jsou
hodnoty 5000 \(Bq.m^{-2}\) a 3 \(MBq.m^{-2}\).
Vedle vrstvy referenčních úrovní je výstupem modelu vrstva překročení
hygienického limitu kontaminace biomasy. Rastrová vrstva nese hodnoty 0 pro
pixely, ve kterých je zjištěna hodnota úrovně hmotnostní kontaminace biomasy
menší než stanovená hodnota v uživatelském rozhraní Urban Green SARCA.
Hodnoty přesahující stanovený hygienický limit jsou zařazeny do kategorie 1.
V modulu Urban Green SARCA je přednastavena hodnota 1000 \((Bq.kg^{-1})\),
která odpovídá nejvyšší přípustné úrovni radioaktivní kontaminace potravin pro
radiačně mimořádné situace podle vyhlášky 389/2010 Sb. o radiační ochraně
(Vyhláška 389/2012 Sb. o radiační ochraně, 2012).


\subsection{Přehled použité literatury}
\label{\detokenize{description:prehled-pouzite-literatury}}
\sphinxstyleemphasis{Muller, H., Prohl, G., 1993. Ecosys\sphinxhyphen{}87: A dynamic model for assessing
radiological consequences of nuclear accidents. Health Phys. 64, 232\textendash{}252.}

\sphinxstyleemphasis{Pröhl, G., 2003. Radioactivity in the terestrial environment, in: Scott, E.M.
(Ed.), Modelling Radioactivity in the Environment. Elsevier, Amsterdam;
Boston, pp. 87\textendash{}108.}

\sphinxstyleemphasis{Rouse Jr, J., Haas, R., Schell, J., Deering, D., 1973. Monitoring vegetation
systems in the Great Plains with ERTS In Third Earth Resources Technology
Satellite\sphinxhyphen{}1, in: Third Earth Resources Technology Satellite\sphinxhyphen{}1 Symposium: The
Proceedings of a Symposium Held by Goddard Space Flight Center at Washington,
DC on December 10\sphinxhyphen{}14, 1973: Prepared at Goddard Space Flight Center.
Scientific and Technical Information Office, NASA, pp. 309\textendash{}317.}

\sphinxstyleemphasis{Vyhláška 389/2012 Sb. o radiační ochraně, 2012.}


\section{Documentation}
\label{\detokenize{libs:documentation}}\label{\detokenize{libs::doc}}

\subsection{Modul activity\_decay}
\label{\detokenize{libs:modul-activity-decay}}
Čekali opakovat tentokrát. Osvobozující, škytla mladou uraženě energický
spáchaná. Mnou nemá úhlů kolo s řezat hrůzy hořkostí že krvavé nedošlo
svlékněte babu. Per pasu tu takzvanou hafáček, místná němu mu hluchý kříž
okamžik třásly k ptá vina oč hygienickým vášeň řeč on mé výslechů. Slušné
jenom působilo míjí vševědoucí všech chvílích spíži bys nějakém cennou ‚milý
zklamáním vy ruce úředník drahocenný. No špínu i po páčit milosrdní nechutí
dovedu. Líp tu ti aut neznámými prožil odkud kamením?

\phantomsection\label{\detokenize{libs:module-activity_decay}}\index{module@\spxentry{module}!activity\_decay@\spxentry{activity\_decay}}\index{activity\_decay@\spxentry{activity\_decay}!module@\spxentry{module}}\index{activityDecay() (in module activity\_decay)@\spxentry{activityDecay()}\spxextra{in module activity\_decay}}

\begin{fulllineitems}
\phantomsection\label{\detokenize{libs:activity_decay.activityDecay}}\pysiglinewithargsret{\sphinxcode{\sphinxupquote{activity\_decay.}}\sphinxbfcode{\sphinxupquote{activityDecay}}}{\emph{\DUrole{n}{A\_0}}, \emph{\DUrole{n}{days}}}{}
Activity decay (Bq/m2) on basis of relationship:
A = A\_0 * exp(\sphinxhyphen{}lambda * t)

Inputs:
:param A\_0: Activity on the start of the time period (Bq/m2)
:type A\_0: Numpy array (float)
:param days: Number of days in time period.
:type days: float

Returns:
:return A: Activity on the end of the time period (Bq/m2)
:rtype A: Numpy array (float)

\end{fulllineitems}



\subsection{Modul hydrIO}
\label{\detokenize{libs:modul-hydrio}}
Musili holiči začátkem černý půjčte? Potrpí zazvonění slečnám. Hm porca cín
kvašení vraty trestat. S nepořádnou bohatá akt styků darovat petrolejku.
Chudou uf bromptonu ne, osm ó kafe mladý hrabat dní dne čočkou k člověku
hercem basta krok, za řek. Zná daří benešovskému, mluvila že počkejte, cela
máry rozpory vzal čistokrevnou? Tmě ji arch tej analýzou.

\phantomsection\label{\detokenize{libs:module-hydrIO}}\index{module@\spxentry{module}!hydrIO@\spxentry{hydrIO}}\index{hydrIO@\spxentry{hydrIO}!module@\spxentry{module}}\index{arrayToRast() (in module hydrIO)@\spxentry{arrayToRast()}\spxextra{in module hydrIO}}

\begin{fulllineitems}
\phantomsection\label{\detokenize{libs:hydrIO.arrayToRast}}\pysiglinewithargsret{\sphinxcode{\sphinxupquote{hydrIO.}}\sphinxbfcode{\sphinxupquote{arrayToRast}}}{\emph{\DUrole{n}{arrays}}, \emph{\DUrole{n}{names}}, \emph{\DUrole{n}{prj}}, \emph{\DUrole{n}{gtransf}}, \emph{\DUrole{n}{EPSG}}, \emph{\DUrole{n}{out\_folder}}, \emph{\DUrole{n}{out\_file\_name}\DUrole{o}{=}\DUrole{default_value}{None}}, \emph{\DUrole{n}{driver\_name}\DUrole{o}{=}\DUrole{default_value}{\textquotesingle{}GTiff\textquotesingle{}}}, \emph{\DUrole{n}{multiband}\DUrole{o}{=}\DUrole{default_value}{False}}}{}
Export numpy 2D arrays to multiband or singleband raster files. Following
common raster formats are accepted for export:
\begin{itemize}
\item {} 
ENVI .hdr labeled raster format

\item {} 
Erdas Imagine (.img) raster format

\item {} 
Idrisi raster format (.rst)

\item {} 
TIFF / BigTIFF / GeoTIFF (.tif) raster format

\item {} 
PCI Geomatics Database File (.pix) raster format

\end{itemize}

\sphinxstylestrong{Required inputs}
:param arrays: Numpy array or list of arrays for export to raster.
:type arrays: numpy.ndarray or list of numpy.ndarray
:param names: Name or list of names of the exported bands (in case
\begin{quote}

of multiband raster) or particular rasters (in case of singleband
rasters).
\end{quote}
\begin{quote}\begin{description}
\item[{Parameters}] \leavevmode\begin{itemize}
\item {} 
\sphinxstyleliteralstrong{\sphinxupquote{prj}} (\sphinxstyleliteralemphasis{\sphinxupquote{str}}) \textendash{} Projection information of the exported raster (dataset).

\item {} 
\sphinxstyleliteralstrong{\sphinxupquote{gtransf}} (\sphinxstyleliteralemphasis{\sphinxupquote{tuple}}) \textendash{} The affine transformation coefficients.

\item {} 
\sphinxstyleliteralstrong{\sphinxupquote{EPSG}} (\sphinxstyleliteralemphasis{\sphinxupquote{int}}) \textendash{} EPSG Geodetic Parameter Set code.

\item {} 
\sphinxstyleliteralstrong{\sphinxupquote{out\_folder}} (\sphinxstyleliteralemphasis{\sphinxupquote{str}}) \textendash{} Path to folder where the raster(s) will be created.

\end{itemize}

\end{description}\end{quote}

\sphinxstylestrong{Optional inputs}
:param driver\_name: GDAL driver. ‘GTiff’ is default.
:type driver\_name: str
:param out\_file\_name: Name of exported multiband raster. Default is None.
:type out\_file\_name: str
:param multiband: Option of multiband raster creation. Default is False.
:type multiband: bool

\sphinxstylestrong{Returns}
:returns: Raster singleband or multiband file(s)
:rtype: raster

\end{fulllineitems}

\index{rasterToArray() (in module hydrIO)@\spxentry{rasterToArray()}\spxextra{in module hydrIO}}

\begin{fulllineitems}
\phantomsection\label{\detokenize{libs:hydrIO.rasterToArray}}\pysiglinewithargsret{\sphinxcode{\sphinxupquote{hydrIO.}}\sphinxbfcode{\sphinxupquote{rasterToArray}}}{\emph{\DUrole{n}{layer}}}{}
Conversion of raster layer to numpy array.
:param layer: Path to raster layer.
:type layer: str
\begin{quote}\begin{description}
\item[{Returns}] \leavevmode
raster file converted to numpy array

\end{description}\end{quote}

\end{fulllineitems}

\index{readGeo() (in module hydrIO)@\spxentry{readGeo()}\spxextra{in module hydrIO}}

\begin{fulllineitems}
\phantomsection\label{\detokenize{libs:hydrIO.readGeo}}\pysiglinewithargsret{\sphinxcode{\sphinxupquote{hydrIO.}}\sphinxbfcode{\sphinxupquote{readGeo}}}{\emph{\DUrole{n}{rast}}}{}
Reading important geographical information from raster using GDAL.
\begin{quote}\begin{description}
\item[{Parameters}] \leavevmode
\sphinxstyleliteralstrong{\sphinxupquote{rast}} (\sphinxstyleliteralemphasis{\sphinxupquote{str}}) \textendash{} Path to raster file in GDAL accepted format.

\item[{Return gtransf}] \leavevmode
The affine transformation coefficients.

\item[{Rtype gtransf}] \leavevmode
tuple

\item[{Return prj}] \leavevmode
Projection information of the raster (dataset).

\item[{Rtype prj}] \leavevmode
str

\item[{Return xSize}] \leavevmode
Pixel width (m).

\item[{Rtype xSize}] \leavevmode
float

\item[{Return ySize}] \leavevmode
Pixel heigth (m)

\item[{Rtype ySize}] \leavevmode
float

\item[{Return EPSG}] \leavevmode
EPSG Geodetic Parameter Set code.

\item[{Rtype EPSG}] \leavevmode
int

\end{description}\end{quote}

\end{fulllineitems}

\index{readLatLong() (in module hydrIO)@\spxentry{readLatLong()}\spxextra{in module hydrIO}}

\begin{fulllineitems}
\phantomsection\label{\detokenize{libs:hydrIO.readLatLong}}\pysiglinewithargsret{\sphinxcode{\sphinxupquote{hydrIO.}}\sphinxbfcode{\sphinxupquote{readLatLong}}}{\emph{\DUrole{n}{rast\_path}}}{}
Automatic setting of the lyrs coordinates according to the
projection of NIR band in to the form.

\end{fulllineitems}



\subsection{Modul SARCA\_lib}
\label{\detokenize{libs:modul-sarca-lib}}
Musili holiči začátkem černý půjčte? Potrpí zazvonění slečnám. Hm porca cín
kvašení vraty trestat. S nepořádnou bohatá akt styků darovat petrolejku.
Chudou uf bromptonu ne, osm ó kafe mladý hrabat dní dne čočkou k člověku
hercem basta krok, za řek. Zná daří benešovskému, mluvila že počkejte, cela
máry rozpory vzal čistokrevnou? Tmě ji arch tej analýzou.

\phantomsection\label{\detokenize{libs:module-SARCA_lib}}\index{module@\spxentry{module}!SARCA\_lib@\spxentry{SARCA\_lib}}\index{SARCA\_lib@\spxentry{SARCA\_lib}!module@\spxentry{module}}\index{SARCALib (class in SARCA\_lib)@\spxentry{SARCALib}\spxextra{class in SARCA\_lib}}

\begin{fulllineitems}
\phantomsection\label{\detokenize{libs:SARCA_lib.SARCALib}}\pysigline{\sphinxbfcode{\sphinxupquote{class }}\sphinxcode{\sphinxupquote{SARCA\_lib.}}\sphinxbfcode{\sphinxupquote{SARCALib}}}
Library for calculation of crops growth parameters and radioactivity
contamination of crops
\index{calculateGrowthCoefs() (SARCA\_lib.SARCALib method)@\spxentry{calculateGrowthCoefs()}\spxextra{SARCA\_lib.SARCALib method}}

\begin{fulllineitems}
\phantomsection\label{\detokenize{libs:SARCA_lib.SARCALib.calculateGrowthCoefs}}\pysiglinewithargsret{\sphinxbfcode{\sphinxupquote{calculateGrowthCoefs}}}{\emph{\DUrole{n}{dw\_max}}, \emph{\DUrole{n}{dw\_min}\DUrole{o}{=}\DUrole{default_value}{0.1}}}{}
Calculate default values of growth curve parameters \sphinxhyphen{} slope
(m) and intercept (n).
\begin{quote}\begin{description}
\item[{Parameters}] \leavevmode
\sphinxstyleliteralstrong{\sphinxupquote{dw\_max}} \textendash{} Maximal dry mass of particular crop :math:{\color{red}\bfseries{}\textasciigrave{}}(

\end{description}\end{quote}

t.ha\textasciicircum{}\{\sphinxhyphen{}1\})\textasciigrave{}
:param dw\_min: Minimal dry mass of particular crop \((
t.ha^{-1})\). Default value is 0.1 \((
t.ha^{-1})\).
\begin{quote}\begin{description}
\item[{Returns}] \leavevmode
Slope of growth curve

\item[{Returns}] \leavevmode
Intercept of growth curve

\end{description}\end{quote}

\end{fulllineitems}


\end{fulllineitems}



\subsection{Modul usle}
\label{\detokenize{libs:modul-usle}}
Musili holiči začátkem černý půjčte? Potrpí zazvonění slečnám. Hm porca cín
kvašení vraty trestat. S nepořádnou bohatá akt styků darovat petrolejku.
Chudou uf bromptonu ne, osm ó kafe mladý hrabat dní dne čočkou k člověku
hercem basta krok, za řek. Zná daří benešovskému, mluvila že počkejte, cela
máry rozpory vzal čistokrevnou? Tmě ji arch tej analýzou.

\phantomsection\label{\detokenize{libs:module-usle}}\index{module@\spxentry{module}!usle@\spxentry{usle}}\index{usle@\spxentry{usle}!module@\spxentry{module}}\index{RadUSLE (class in usle)@\spxentry{RadUSLE}\spxextra{class in usle}}

\begin{fulllineitems}
\phantomsection\label{\detokenize{libs:usle.RadUSLE}}\pysigline{\sphinxbfcode{\sphinxupquote{class }}\sphinxcode{\sphinxupquote{usle.}}\sphinxbfcode{\sphinxupquote{RadUSLE}}}
Calculation of USLE
\index{fC() (usle.RadUSLE method)@\spxentry{fC()}\spxextra{usle.RadUSLE method}}

\begin{fulllineitems}
\phantomsection\label{\detokenize{libs:usle.RadUSLE.fC}}\pysiglinewithargsret{\sphinxbfcode{\sphinxupquote{fC}}}{\emph{\DUrole{n}{crops}}, \emph{\DUrole{n}{c\_values}}}{}
Crop factor of USLE.
\begin{quote}\begin{description}
\item[{Parameters}] \leavevmode\begin{itemize}
\item {} 
\sphinxstyleliteralstrong{\sphinxupquote{crops}} (\sphinxstyleliteralemphasis{\sphinxupquote{Numpy array}}\sphinxstyleliteralemphasis{\sphinxupquote{ (}}\sphinxstyleliteralemphasis{\sphinxupquote{int}}\sphinxstyleliteralemphasis{\sphinxupquote{)}}) \textendash{} 
Layer with codes of C factor for particular crops.
The codes are following:
\begin{quote}

Crops:
1: winter wheat
2: spring wheat
3: winter rye
4: spring barley
5: winter barley
6: oat
7: maize (corn)
8: legumes
9: early potatoes
10: late potatoes
11: meadows
12: hoppers
13: winter rape
14:     sunflower
15: poppy
16: another oilseeds
17:     maize (silage)
18: another one\sphinxhyphen{}year\sphinxhyphen{}olds fodder crops
19: another perenial fodder crops
20:     vegetables
21: orchards
22: forests
23: municipalities
24: bare soil
\end{quote}


\item {} 
\sphinxstyleliteralstrong{\sphinxupquote{c\_values}} \textendash{} C values corresponding to crop categories

\end{itemize}

\end{description}\end{quote}

are set according to Janeček et al. (2007)
:type c\_values: list

Returns:
:return: Matrix of C values.
:rtype: Numpy array (float)

\end{fulllineitems}

\index{fK() (usle.RadUSLE method)@\spxentry{fK()}\spxextra{usle.RadUSLE method}}

\begin{fulllineitems}
\phantomsection\label{\detokenize{libs:usle.RadUSLE.fK}}\pysiglinewithargsret{\sphinxbfcode{\sphinxupquote{fK}}}{\emph{\DUrole{n}{HPJ}}, \emph{\DUrole{n}{k\_values}}}{}
K erodibility factor of USLE.

Inputs:
:param HPJ: Numpy array with Main soil units codes (HPJ \sphinxhyphen{} hlavní
\begin{quote}

půdní jednotky, after Janeček et al. 2007).
\end{quote}
\begin{quote}\begin{description}
\item[{Parameters}] \leavevmode
\sphinxstyleliteralstrong{\sphinxupquote{k\_values}} \textendash{} K values corresponding to Main soil units

\end{description}\end{quote}

codes are set according to Janeček et al. (2007)
:type k\_values: list

Returns:
:return K\_matrix: Matrix of K values.
:rtype K\_matrix: Numpy array, float

\end{fulllineitems}

\index{fLS() (usle.RadUSLE method)@\spxentry{fLS()}\spxextra{usle.RadUSLE method}}

\begin{fulllineitems}
\phantomsection\label{\detokenize{libs:usle.RadUSLE.fLS}}\pysiglinewithargsret{\sphinxbfcode{\sphinxupquote{fLS}}}{\emph{\DUrole{n}{flowac}}, \emph{\DUrole{n}{slope}}, \emph{\DUrole{n}{xSize}\DUrole{o}{=}\DUrole{default_value}{1}}, \emph{\DUrole{n}{m}\DUrole{o}{=}\DUrole{default_value}{0.4}}, \emph{\DUrole{n}{n}\DUrole{o}{=}\DUrole{default_value}{1.4}}}{}
Combined factor of slope length and slope steepness factor
of USLE.

Inputs:
:param flowac: Flow accumulation probability grid.
:type flowac: Numpy array
:param slope: Slope grid (degrees)
:type slope: Numpy array
:param xSize: Size of pixel (m)
:type xSize: float
:param m: Exponent representing the Rill\sphinxhyphen{}to\sphinxhyphen{}Interrill Ratio.
\begin{quote}

Default m = 0.4
\end{quote}
\begin{quote}\begin{description}
\item[{Parameters}] \leavevmode
\sphinxstyleliteralstrong{\sphinxupquote{n}} (\sphinxstyleliteralemphasis{\sphinxupquote{float}}) \textendash{} Constant. Default n = 1.4

\end{description}\end{quote}

Returns:
:return LS: Combined factor of slope length and slope steepness
\begin{quote}

factor of USLE
\end{quote}
\begin{quote}\begin{description}
\item[{Rtype LS}] \leavevmode
Numpy array

\end{description}\end{quote}

\end{fulllineitems}

\index{fR() (usle.RadUSLE method)@\spxentry{fR()}\spxextra{usle.RadUSLE method}}

\begin{fulllineitems}
\phantomsection\label{\detokenize{libs:usle.RadUSLE.fR}}\pysiglinewithargsret{\sphinxbfcode{\sphinxupquote{fR}}}{\emph{\DUrole{n}{R\_const}\DUrole{o}{=}\DUrole{default_value}{40}}, \emph{\DUrole{n}{month\_perc}\DUrole{o}{=}\DUrole{default_value}{32.2}}}{}
R erosivity factor of USLE for monthly data.

Inputs:
:param R\_const: Constant year value of R factor for particular
\begin{quote}

area (MJ/ha)
\end{quote}
\begin{quote}\begin{description}
\item[{Parameters}] \leavevmode
\sphinxstyleliteralstrong{\sphinxupquote{month\_perc}} (\sphinxstyleliteralemphasis{\sphinxupquote{float}}) \textendash{} Percentage of R for particular months.

\end{description}\end{quote}

Returns:
:return R: R value for particular month.
:rtype R: float

\end{fulllineitems}

\index{slope() (usle.RadUSLE method)@\spxentry{slope()}\spxextra{usle.RadUSLE method}}

\begin{fulllineitems}
\phantomsection\label{\detokenize{libs:usle.RadUSLE.slope}}\pysiglinewithargsret{\sphinxbfcode{\sphinxupquote{slope}}}{\emph{\DUrole{n}{dmt}}, \emph{\DUrole{n}{xSize}\DUrole{o}{=}\DUrole{default_value}{1}}, \emph{\DUrole{n}{ySize}\DUrole{o}{=}\DUrole{default_value}{1}}}{}
Function calculates slope of terrain (DMT) in degrees

Inputs:
:param dmt: Digital elevation model.
:type dmt: Numpy array
:param xSize: Size of pixel in x axis (m)
:type xSze: float
:param ySize: Size of pixel in y axis (m)
:type ySize: float

Returns: float
:return slope: Slope of terrain (DMT) in degrees
:rtype slope: Numpy array

\end{fulllineitems}

\index{usle() (usle.RadUSLE method)@\spxentry{usle()}\spxextra{usle.RadUSLE method}}

\begin{fulllineitems}
\phantomsection\label{\detokenize{libs:usle.RadUSLE.usle}}\pysiglinewithargsret{\sphinxbfcode{\sphinxupquote{usle}}}{\emph{\DUrole{n}{R}}, \emph{\DUrole{n}{K}}, \emph{\DUrole{n}{LS}}, \emph{\DUrole{n}{C}}, \emph{\DUrole{n}{P}\DUrole{o}{=}\DUrole{default_value}{1}}}{}
Universal Soil Loss Equation.

Inputs:
:param R: R erosivity factor of USLE.
:type R: Numpy array (float)
:param K: K erodibility factor of USLE.
:type K: Numpy array (float)
:param LS: Combined factor of slope length and slope steepness
\begin{quote}

factor of USLE.
\end{quote}
\begin{quote}\begin{description}
\item[{Parameters}] \leavevmode\begin{itemize}
\item {} 
\sphinxstyleliteralstrong{\sphinxupquote{C}} (\sphinxstyleliteralemphasis{\sphinxupquote{Numpy array}}\sphinxstyleliteralemphasis{\sphinxupquote{ (}}\sphinxstyleliteralemphasis{\sphinxupquote{float}}\sphinxstyleliteralemphasis{\sphinxupquote{)}}) \textendash{} C cover management impact factor of USLE.

\item {} 
\sphinxstyleliteralstrong{\sphinxupquote{P}} (\sphinxstyleliteralemphasis{\sphinxupquote{Numpy array}}\sphinxstyleliteralemphasis{\sphinxupquote{ (}}\sphinxstyleliteralemphasis{\sphinxupquote{float}}\sphinxstyleliteralemphasis{\sphinxupquote{) or }}\sphinxstyleliteralemphasis{\sphinxupquote{float.}}) \textendash{} P support practices factor of USLE.

\end{itemize}

\end{description}\end{quote}

Returns:
:return: Spatial and temporal soil loss (t/ha). Here,
\begin{quote}

the equation is calculated for monthly data.
\end{quote}
\begin{quote}\begin{description}
\item[{Return type}] \leavevmode
Numpy array (float)

\end{description}\end{quote}

\end{fulllineitems}


\end{fulllineitems}



\subsection{Modul waterflow}
\label{\detokenize{libs:modul-waterflow}}
Musili holiči začátkem černý půjčte? Potrpí zazvonění slečnám. Hm porca cín
kvašení vraty trestat. S nepořádnou bohatá akt styků darovat petrolejku.
Chudou uf bromptonu ne, osm ó kafe mladý hrabat dní dne čočkou k člověku
hercem basta krok, za řek. Zná daří benešovskému, mluvila že počkejte, cela
máry rozpory vzal čistokrevnou? Tmě ji arch tej analýzou.

\phantomsection\label{\detokenize{libs:module-waterflow}}\index{module@\spxentry{module}!waterflow@\spxentry{waterflow}}\index{waterflow@\spxentry{waterflow}!module@\spxentry{module}}\index{WaterBalance (class in waterflow)@\spxentry{WaterBalance}\spxextra{class in waterflow}}

\begin{fulllineitems}
\phantomsection\label{\detokenize{libs:waterflow.WaterBalance}}\pysigline{\sphinxbfcode{\sphinxupquote{class }}\sphinxcode{\sphinxupquote{waterflow.}}\sphinxbfcode{\sphinxupquote{WaterBalance}}}
Module for calculation of the hydrological features of the area of
interest.
\index{airTemperToGrid() (waterflow.WaterBalance method)@\spxentry{airTemperToGrid()}\spxextra{waterflow.WaterBalance method}}

\begin{fulllineitems}
\phantomsection\label{\detokenize{libs:waterflow.WaterBalance.airTemperToGrid}}\pysiglinewithargsret{\sphinxbfcode{\sphinxupquote{airTemperToGrid}}}{\emph{\DUrole{n}{tm\_list}}, \emph{\DUrole{n}{dmt}}, \emph{\DUrole{n}{altitude}}, \emph{\DUrole{n}{adiab}\DUrole{o}{=}\DUrole{default_value}{0.65}}}{}
Calculation of spatial temperature distribution in accordance
to altitude (DEM). The function provides list (Numpy array) of air
temperature arrays corresponding to list of measured temperature
data.

Inputs:
:param tm\_list: List of air temperatures measured on a meteostation.
:type tm\_list: list
:param dmt: Digital elevation model.
:type dmt: Numpy array
:param altitude: Altitude of temperature measurement.
:type altitude: float
:param adiab: Adiabatic change of temeperature with altitude
\begin{quote}

per 100 m. Default value is 0.65 °C/100 m.
\end{quote}
\begin{quote}\begin{description}
\end{description}\end{quote}

Returns:
:return tm\_grids: List of air temperature grids.
:rtype tm\_grids: Numpy array

\end{fulllineitems}

\index{evapoActual() (waterflow.WaterBalance method)@\spxentry{evapoActual()}\spxextra{waterflow.WaterBalance method}}

\begin{fulllineitems}
\phantomsection\label{\detokenize{libs:waterflow.WaterBalance.evapoActual}}\pysiglinewithargsret{\sphinxbfcode{\sphinxupquote{evapoActual}}}{\emph{\DUrole{n}{ETp}}, \emph{\DUrole{n}{precip}}}{}
Actual evapotranspiration calculated according to Ol’dekop (1911,
cited after Brutsaert (1992) and Xiong and Guo 
(1999; doi.org/10.1016/S0022\sphinxhyphen{}1694(98)00297\sphinxhyphen{}2)

Inputs:
:param ETp: Potential monthly evapotranspiration according
to Thornthwaite (1984), mm. List of monthly values for the year.
:type ETp: list
:param precip: Mean monthly precipitation throughout the year (mm).
:type precip: list

Returns
:return ETa: Actual monthly evapotranspiration throughout the year (mm) 
:rtype ETa: Numpy array

\end{fulllineitems}

\index{evapoPot() (waterflow.WaterBalance method)@\spxentry{evapoPot()}\spxextra{waterflow.WaterBalance method}}

\begin{fulllineitems}
\phantomsection\label{\detokenize{libs:waterflow.WaterBalance.evapoPot}}\pysiglinewithargsret{\sphinxbfcode{\sphinxupquote{evapoPot}}}{\emph{\DUrole{n}{tm\_grids}}, \emph{\DUrole{n}{lat}\DUrole{o}{=}\DUrole{default_value}{49.1797903}}}{}
Potential monthly ET According to Thornthwaite 1948. Script
calculates ETpot for the whole year \sphinxhyphen{} for each month.

Inputs:
:param tm\_grids: List of monthly mean air temperatures during
the year (degree of Celsius) \sphinxhyphen{} temperature normals
:type tm\_grids: list
:param lat: Earth latitude (UTM) in decimal degrees
:type lat: float

Returns
:return ET\_pot: Potential monthly evapotranspiration according
to Thornthwaite (1984), mm. List of monthly values for the year.
:rtype ETpot\_list: Numpy array

\end{fulllineitems}

\index{flowAccProb() (waterflow.WaterBalance method)@\spxentry{flowAccProb()}\spxextra{waterflow.WaterBalance method}}

\begin{fulllineitems}
\phantomsection\label{\detokenize{libs:waterflow.WaterBalance.flowAccProb}}\pysiglinewithargsret{\sphinxbfcode{\sphinxupquote{flowAccProb}}}{\emph{\DUrole{n}{dmt}}, \emph{\DUrole{n}{xsize}\DUrole{o}{=}\DUrole{default_value}{1.0}}, \emph{\DUrole{n}{ysize}\DUrole{o}{=}\DUrole{default_value}{1.0}}, \emph{\DUrole{n}{rs}\DUrole{o}{=}\DUrole{default_value}{None}}}{}
Calculation of flow accumulation probability layer according
to digital elevation model. Probability of flow direction within
DEM is calculated on basis of shape of the surface (outflow
changes linearly with changing angle between neighbour pixels) 
and surface resistance for surface runoff (rel.).
The function was inspired by Multipath\sphinxhyphen{}Flow\sphinxhyphen{}Accumulation developed
by Alex Stum: \sphinxurl{https://github.com/StumWhere/Multipath-Flow-Accumulation.git}
\begin{quote}\begin{description}
\item[{Parameters}] \leavevmode\begin{itemize}
\item {} 
\sphinxstyleliteralstrong{\sphinxupquote{dmt}} (\sphinxstyleliteralemphasis{\sphinxupquote{numpy.ndarray}}) \textendash{} Digital elevation model of the surface (m).

\item {} 
\sphinxstyleliteralstrong{\sphinxupquote{xsize}} (\sphinxstyleliteralemphasis{\sphinxupquote{float}}) \textendash{} Size of pixel in x axis (m)

\item {} 
\sphinxstyleliteralstrong{\sphinxupquote{ysize}} (\sphinxstyleliteralemphasis{\sphinxupquote{float}}) \textendash{} Size of pixel in y axis (m)

\item {} 
\sphinxstyleliteralstrong{\sphinxupquote{rs}} (\sphinxstyleliteralemphasis{\sphinxupquote{numpy.ndarray}}) \textendash{} Surface resistance wor surface runoff of water scaled
to interval \textless{}0; 1\textgreater{}, where 0 is no resistance
and 1 is 100\% resistance (no flow). Scaled Mannings n
should be used. Default is None (zero resistance is used).

\end{itemize}

\item[{Return accum}] \leavevmode
Flow accumulation grid.

\item[{Rtype accum}] \leavevmode
numpy.ndarray

\end{description}\end{quote}

\end{fulllineitems}

\index{flowProbab() (waterflow.WaterBalance method)@\spxentry{flowProbab()}\spxextra{waterflow.WaterBalance method}}

\begin{fulllineitems}
\phantomsection\label{\detokenize{libs:waterflow.WaterBalance.flowProbab}}\pysiglinewithargsret{\sphinxbfcode{\sphinxupquote{flowProbab}}}{\emph{\DUrole{n}{win\_dmt}}, \emph{\DUrole{n}{xsize}\DUrole{o}{=}\DUrole{default_value}{1.0}}, \emph{\DUrole{n}{ysize}\DUrole{o}{=}\DUrole{default_value}{1.0}}}{}
Calculation of probability of water flow direction in 3x3 matrix
on basis of elevation data.

Inputs:
:param win\_dmt: 3x3 matrix of elevation model.
:type win\_dmt: numpy array
:param xsize: Size of pixel in x axis (m)
:type xsize: float
:param ysize: Size of pixel in y axis (m)
:type ysize: float

\end{fulllineitems}

\index{interceptWater() (waterflow.WaterBalance method)@\spxentry{interceptWater()}\spxextra{waterflow.WaterBalance method}}

\begin{fulllineitems}
\phantomsection\label{\detokenize{libs:waterflow.WaterBalance.interceptWater}}\pysiglinewithargsret{\sphinxbfcode{\sphinxupquote{interceptWater}}}{\emph{\DUrole{n}{precip}}, \emph{\DUrole{n}{LAI}}, \emph{\DUrole{n}{a}\DUrole{o}{=}\DUrole{default_value}{0.1}}, \emph{\DUrole{n}{b}\DUrole{o}{=}\DUrole{default_value}{0.2}}}{}
Interception of precipitation on the biomass and soil surface 
for monthly precipitation data (mm)

Inputs:
:param precip: Grid of monthly mean precipitation amount (mm)
:type precip: Numpy array
:param LAI: Grid of monthly mean leaf area index (unitless)
:type LAI: Numpy array
:param a: Constant
:type a: float
:param b: Constant
:type b: float

Returns:
:return I: Grid of amount of intercepted water during month (mm)
:rtype I: Numpy array

\end{fulllineitems}

\index{runoffSeparCN() (waterflow.WaterBalance method)@\spxentry{runoffSeparCN()}\spxextra{waterflow.WaterBalance method}}

\begin{fulllineitems}
\phantomsection\label{\detokenize{libs:waterflow.WaterBalance.runoffSeparCN}}\pysiglinewithargsret{\sphinxbfcode{\sphinxupquote{runoffSeparCN}}}{\emph{\DUrole{n}{acc\_precip}}, \emph{\DUrole{n}{ET}}, \emph{\DUrole{n}{ETp}}, \emph{\DUrole{n}{I}}, \emph{\DUrole{n}{CN}\DUrole{o}{=}\DUrole{default_value}{65}}, \emph{\DUrole{n}{a}\DUrole{o}{=}\DUrole{default_value}{0.005}}, \emph{\DUrole{n}{b}\DUrole{o}{=}\DUrole{default_value}{0.005}}, \emph{\DUrole{n}{c}\DUrole{o}{=}\DUrole{default_value}{0.5}}}{}
The script separates precipitation ammount (accumulated) into 
surface runoff, water retention and evapotranspiration for
monthly precipitation data. The method is based on the modified
CN curve method.

Inputs:
:param acc\_precip: Monthly mean of acc\_precipitation (mm)
:type acc\_precip: float
:param ET: Monthly mean evapotranspiration (mm)
:type ET: float
:param ETp: Monthly mean potential evapotranspiration (mm)
:type ETp: float
:param I: Amount of intercepted water during month (mm)
:type I: float
:param CN: Curve number
:type CN: int
:param a: Constant
:type a: float
:param b: Constant
:type b: float
:param c: Constant
:type c: float
:param d: Constant
:type d: float
:param e: Constant
:type e: float

Returns:
:return Rcor: Amount of monthly surface runoff (mm) corrected on ET
:rtype Rcor: float
:return Scor: Amount of retention of water in the soil
\begin{quote}

or subsurface runoff (mm) corrected on ET
\end{quote}
\begin{quote}\begin{description}
\item[{Rtype Scor}] \leavevmode
float

\item[{Return ETa}] \leavevmode
Monthly amount of actual evapotranspiration 
from the surface (mm)

\item[{Rtype ETa}] \leavevmode
float

\end{description}\end{quote}

\end{fulllineitems}

\index{waterFlows() (waterflow.WaterBalance method)@\spxentry{waterFlows()}\spxextra{waterflow.WaterBalance method}}

\begin{fulllineitems}
\phantomsection\label{\detokenize{libs:waterflow.WaterBalance.waterFlows}}\pysiglinewithargsret{\sphinxbfcode{\sphinxupquote{waterFlows}}}{\emph{\DUrole{n}{dmt}}, \emph{\DUrole{n}{precip}}, \emph{\DUrole{n}{CN}}, \emph{\DUrole{n}{LAI}}, \emph{\DUrole{n}{ETp}}, \emph{\DUrole{n}{xsize}\DUrole{o}{=}\DUrole{default_value}{1.0}}, \emph{\DUrole{n}{ysize}\DUrole{o}{=}\DUrole{default_value}{1.0}}, \emph{\DUrole{n}{a}\DUrole{o}{=}\DUrole{default_value}{0.005}}, \emph{\DUrole{n}{b}\DUrole{o}{=}\DUrole{default_value}{0.005}}, \emph{\DUrole{n}{c}\DUrole{o}{=}\DUrole{default_value}{0.5}}, \emph{\DUrole{n}{d}\DUrole{o}{=}\DUrole{default_value}{0.1}}, \emph{\DUrole{n}{e}\DUrole{o}{=}\DUrole{default_value}{0.2}}}{}
Calculation of flow accumulation according to digital elevation model.
Probability of flow direction within DEM is calculated on basis 
of shape of the surface (outflow changes linearly with changing angle
between neighbour pixels) and surface resistance for surface runoff
(rel.).
The function was inspired by Multipath\sphinxhyphen{}Flow\sphinxhyphen{}Accumulation developed
by Alex Stum: \sphinxurl{https://github.com/StumWhere/Multipath-Flow-Accumulation.git}
\begin{quote}\begin{description}
\item[{Parameters}] \leavevmode\begin{itemize}
\item {} 
\sphinxstyleliteralstrong{\sphinxupquote{dmt}} (\sphinxstyleliteralemphasis{\sphinxupquote{numpy.ndarray}}) \textendash{} Digital elevation model of the surface (m).

\item {} 
\sphinxstyleliteralstrong{\sphinxupquote{xsize}} (\sphinxstyleliteralemphasis{\sphinxupquote{float}}) \textendash{} Size of pixel in x axis (m)

\item {} 
\sphinxstyleliteralstrong{\sphinxupquote{ysize}} (\sphinxstyleliteralemphasis{\sphinxupquote{float}}) \textendash{} Size of pixel in y axis (m)

\item {} 
\sphinxstyleliteralstrong{\sphinxupquote{rs}} (\sphinxstyleliteralemphasis{\sphinxupquote{numpy.ndarray}}) \textendash{} Surface resistance wor surface runoff of water scaled
to interval \textless{}0; 1\textgreater{}, where 0 is no resistance
and 1 is 100\% resistance (no flow). Scaled Mannings n
should be used. Default is None (zero resistance is used).

\end{itemize}

\item[{Return accum}] \leavevmode
Flow accumulation grid.

\item[{Rtype accum}] \leavevmode
numpy.ndarray

\end{description}\end{quote}

\end{fulllineitems}


\end{fulllineitems}



\subsection{Modul mdaylight}
\label{\detokenize{libs:modul-mdaylight}}
Musili holiči začátkem černý půjčte? Potrpí zazvonění slečnám. Hm porca cín
kvašení vraty trestat. S nepořádnou bohatá akt styků darovat petrolejku.
Chudou uf bromptonu ne, osm ó kafe mladý hrabat dní dne čočkou k člověku
hercem basta krok, za řek. Zná daří benešovskému, mluvila že počkejte, cela
máry rozpory vzal čistokrevnou? Tmě ji arch tej analýzou.

\phantomsection\label{\detokenize{libs:module-mdaylight}}\index{module@\spxentry{module}!mdaylight@\spxentry{mdaylight}}\index{mdaylight@\spxentry{mdaylight}!module@\spxentry{module}}\index{MonthlyDaylight (class in mdaylight)@\spxentry{MonthlyDaylight}\spxextra{class in mdaylight}}

\begin{fulllineitems}
\phantomsection\label{\detokenize{libs:mdaylight.MonthlyDaylight}}\pysigline{\sphinxbfcode{\sphinxupquote{class }}\sphinxcode{\sphinxupquote{mdaylight.}}\sphinxbfcode{\sphinxupquote{MonthlyDaylight}}}
Calculates list of monthly mean daylight, timezone and length of
dailight for particular day according to geographical position.
\index{dayLength() (mdaylight.MonthlyDaylight method)@\spxentry{dayLength()}\spxextra{mdaylight.MonthlyDaylight method}}

\begin{fulllineitems}
\phantomsection\label{\detokenize{libs:mdaylight.MonthlyDaylight.dayLength}}\pysiglinewithargsret{\sphinxbfcode{\sphinxupquote{dayLength}}}{\emph{\DUrole{n}{nday}\DUrole{o}{=}\DUrole{default_value}{1}}, \emph{\DUrole{n}{lat}\DUrole{o}{=}\DUrole{default_value}{49.1797903}}}{}
Calculation of dalight according to geographic position and day
number (1\sphinxhyphen{}365).

Inputs:
:param nday: Number of day throughout the year (0\sphinxhyphen{}365)
:type nday: int
:param lat: Earth latitude (UTM) in decimal degrees
:type lat: float

Returns
:return: List of mean monthly daylight (hours, decimal)
:rtype: list

\end{fulllineitems}

\index{monthlyDaylights() (mdaylight.MonthlyDaylight method)@\spxentry{monthlyDaylights()}\spxextra{mdaylight.MonthlyDaylight method}}

\begin{fulllineitems}
\phantomsection\label{\detokenize{libs:mdaylight.MonthlyDaylight.monthlyDaylights}}\pysiglinewithargsret{\sphinxbfcode{\sphinxupquote{monthlyDaylights}}}{\emph{\DUrole{n}{lat}\DUrole{o}{=}\DUrole{default_value}{49.1797903}}}{}
Calculation of monthly mean dalight \sphinxhyphen{} potential duration of solar 
radiation

Inputs:
:param lat: Earth latitude (UTM) in decimal degrees
:type lat: float

Returns
:return: List of mean monthly daylight (hours, decimal)
:rtype: list

\end{fulllineitems}


\end{fulllineitems}



\subsection{Modul sowing\_proc}
\label{\detokenize{libs:modul-sowing-proc}}
Musili holiči začátkem černý půjčte? Potrpí zazvonění slečnám. Hm porca cín
kvašení vraty trestat. S nepořádnou bohatá akt styků darovat petrolejku.
Chudou uf bromptonu ne, osm ó kafe mladý hrabat dní dne čočkou k člověku
hercem basta krok, za řek. Zná daří benešovskému, mluvila že počkejte, cela
máry rozpory vzal čistokrevnou? Tmě ji arch tej analýzou.

\phantomsection\label{\detokenize{libs:module-sowing_proc}}\index{module@\spxentry{module}!sowing\_proc@\spxentry{sowing\_proc}}\index{sowing\_proc@\spxentry{sowing\_proc}!module@\spxentry{module}}\index{SowingProcTimeSeries (class in sowing\_proc)@\spxentry{SowingProcTimeSeries}\spxextra{class in sowing\_proc}}

\begin{fulllineitems}
\phantomsection\label{\detokenize{libs:sowing_proc.SowingProcTimeSeries}}\pysigline{\sphinxbfcode{\sphinxupquote{class }}\sphinxcode{\sphinxupquote{sowing\_proc.}}\sphinxbfcode{\sphinxupquote{SowingProcTimeSeries}}}
Class SowingProsTimeSeries is a module for calculation of
sowing procedure time series in monthly step. Calculation is
based on order of crops in the sowing procedure and their
agronomy terms (sowing and harvest)
\index{calcMeadows() (sowing\_proc.SowingProcTimeSeries method)@\spxentry{calcMeadows()}\spxextra{sowing\_proc.SowingProcTimeSeries method}}

\begin{fulllineitems}
\phantomsection\label{\detokenize{libs:sowing_proc.SowingProcTimeSeries.calcMeadows}}\pysiglinewithargsret{\sphinxbfcode{\sphinxupquote{calcMeadows}}}{\emph{\DUrole{n}{meadows\_table}}, \emph{\DUrole{n}{ID\_col\_number}\DUrole{o}{=}\DUrole{default_value}{0}}, \emph{\DUrole{n}{cut\_col\_number}\DUrole{o}{=}\DUrole{default_value}{1}}}{}
Calculation of meadows management (cuts) time series. Output is a
list containing IDs meadows management (cuts).
The time series starts in January and ending on the December with
monthly step.
\begin{quote}\begin{description}
\item[{Parameters}] \leavevmode\begin{itemize}
\item {} 
\sphinxstyleliteralstrong{\sphinxupquote{meadows\_table}} \textendash{} 

\item {} 
\sphinxstyleliteralstrong{\sphinxupquote{ID\_col\_number}} \textendash{} 

\item {} 
\sphinxstyleliteralstrong{\sphinxupquote{cut\_col\_number}} \textendash{} 

\end{itemize}

\item[{Returns}] \leavevmode


\end{description}\end{quote}

\end{fulllineitems}

\index{calcSowingProc() (sowing\_proc.SowingProcTimeSeries method)@\spxentry{calcSowingProc()}\spxextra{sowing\_proc.SowingProcTimeSeries method}}

\begin{fulllineitems}
\phantomsection\label{\detokenize{libs:sowing_proc.SowingProcTimeSeries.calcSowingProc}}\pysiglinewithargsret{\sphinxbfcode{\sphinxupquote{calcSowingProc}}}{\emph{\DUrole{n}{crops\_table}}, \emph{\DUrole{n}{ID\_col\_number}\DUrole{o}{=}\DUrole{default_value}{0}}, \emph{\DUrole{n}{sowing\_col\_number}\DUrole{o}{=}\DUrole{default_value}{1}}, \emph{\DUrole{n}{harvest\_col\_number}\DUrole{o}{=}\DUrole{default_value}{2}}}{}
Calculation of sowing procedure time series for crops
included in the crops\_table. Output is a list containing IDs
of crops in sowing procedure.
The time series starts in January of the first year and
ending in the December of the last year of the sowing
procedure. The gaps between crops are highlighted as bare
soil with ID = 999.
\begin{quote}\begin{description}
\item[{Parameters}] \leavevmode
\sphinxstyleliteralstrong{\sphinxupquote{crops\_table}} \textendash{} Pandas dataframe containing information

\end{description}\end{quote}

about crops \sphinxhyphen{} ID of crop, term of sowing and term of harvest
in order of the sowing procedure. Terms of sowing and harvest
are numbered according to months: 1 \sphinxhyphen{} January, 2 \sphinxhyphen{} February etc.
:param ID\_col\_number: crops\_table column with IDs index
:param sowing\_col\_number: crops\_table column with sowing date
index
:param harvest\_col\_number: crops\_table column with harvest date
index
\begin{quote}\begin{description}
\item[{Returns}] \leavevmode
List containing IDs of crops in sowing procedure.

\end{description}\end{quote}

The time series starts in January of the first year and
ending in the December of the last year of the sowing
procedure.

\end{fulllineitems}

\index{predictSowingProc() (sowing\_proc.SowingProcTimeSeries method)@\spxentry{predictSowingProc()}\spxextra{sowing\_proc.SowingProcTimeSeries method}}

\begin{fulllineitems}
\phantomsection\label{\detokenize{libs:sowing_proc.SowingProcTimeSeries.predictSowingProc}}\pysiglinewithargsret{\sphinxbfcode{\sphinxupquote{predictSowingProc}}}{\emph{\DUrole{n}{sow\_time\_series}}, \emph{\DUrole{n}{predict\_months}\DUrole{o}{=}\DUrole{default_value}{12}}, \emph{\DUrole{n}{start}\DUrole{o}{=}\DUrole{default_value}{0}}}{}~\begin{quote}\begin{description}
\item[{Parameters}] \leavevmode\begin{itemize}
\item {} 
\sphinxstyleliteralstrong{\sphinxupquote{sow\_time\_series}} \textendash{} 

\item {} 
\sphinxstyleliteralstrong{\sphinxupquote{predict\_months}} \textendash{} 

\item {} 
\sphinxstyleliteralstrong{\sphinxupquote{start}} \textendash{} 

\end{itemize}

\item[{Returns}] \leavevmode


\end{description}\end{quote}

\end{fulllineitems}


\end{fulllineitems}



\section{Poděkování}
\label{\detokenize{ackn:podekovani}}\label{\detokenize{ackn::doc}}
Vývoj programu RadAgro for QGIS byl podpořen z projektu Ministerstva  vnitra
České republiky VI20172020098.

\noindent\sphinxincludegraphics[width=4cm]{{MV_CR}.png}


\chapter{Indexy a tabulky}
\label{\detokenize{index:indexy-a-tabulky}}\begin{itemize}
\item {} 
\DUrole{xref,std,std-ref}{genindex}

\item {} 
\DUrole{xref,std,std-ref}{modindex}

\item {} 
\DUrole{xref,std,std-ref}{search}

\end{itemize}


\renewcommand{\indexname}{Python Module Index}
\begin{sphinxtheindex}
\let\bigletter\sphinxstyleindexlettergroup
\bigletter{a}
\item\relax\sphinxstyleindexentry{activity\_decay}\sphinxstyleindexpageref{libs:\detokenize{module-activity_decay}}
\indexspace
\bigletter{h}
\item\relax\sphinxstyleindexentry{hydrIO}\sphinxstyleindexpageref{libs:\detokenize{module-hydrIO}}
\indexspace
\bigletter{m}
\item\relax\sphinxstyleindexentry{mdaylight}\sphinxstyleindexpageref{libs:\detokenize{module-mdaylight}}
\indexspace
\bigletter{s}
\item\relax\sphinxstyleindexentry{SARCA\_lib}\sphinxstyleindexpageref{libs:\detokenize{module-SARCA_lib}}
\item\relax\sphinxstyleindexentry{sowing\_proc}\sphinxstyleindexpageref{libs:\detokenize{module-sowing_proc}}
\indexspace
\bigletter{u}
\item\relax\sphinxstyleindexentry{usle}\sphinxstyleindexpageref{libs:\detokenize{module-usle}}
\indexspace
\bigletter{w}
\item\relax\sphinxstyleindexentry{waterflow}\sphinxstyleindexpageref{libs:\detokenize{module-waterflow}}
\end{sphinxtheindex}

\renewcommand{\indexname}{Index}
\printindex
\end{document}